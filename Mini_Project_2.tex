Abstract (100-250 words) Summarize the project task and your most important findings.


Introduction (5+ sentences) Summarize the project task, the dataset, and your most important
findings. This should be similar to the abstract but more detailed.


Related work (4+ sentences) Summarize previous literature related to the sentiment classification
problem.


Dataset and setup (3+ sentences) Very briefly describe the dataset and any basic data
pre-processing methods that are common to all your approaches (e.g., tokenizing). Note : You do
not need to explicitly verify that the data satisfies the i.i.d. assumption (or any of the other formal
assumptions for linear classification).


Proposed approach (7+ sentences ) Briefly describe the different models you implemented/
compared and the features you designed, providing citations as necessary. If you use or build
upon an existing model based on previously published work, it is essential that you properly
cite and acknowledge this previous work. Discuss algorithm selection and implementation. Include
any decisions about training/validation split, regularization strategies, any optimization tricks,
setting hyper-parameters, etc. It is not necessary to provide detailed derivations for the models
you use, but you should provide at least few sentences of background (and motivation) for each
model.
3
Results (7+ sentences, possibly with figures or tables) Provide results on the different
models you implemented (e.g., accuracy on the validation set, runtimes). You should report your
leaderboard test set accuracy in this section, but most of your results should be on your validation
set (or from cross validation).

Discussion and Conclusion (3+ sentences) Summarize the key takeaways from the project
and possibly directions for future investigation.

Statement of Contributions (1-3 sentences) State the breakdown of the workload.